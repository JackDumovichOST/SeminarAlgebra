%
% semesterplan.tex -- Semesterplan
%
% (c) 2021 Prof Dr Andreas Müller, OST Ostschweizer Fachhochschule
%
\bgroup
\definecolor{darkgreen}{rgb}{0,0.6,0}
\definecolor{darkred}{rgb}{0.8,0,0}
\def\w#1#2{#1{\tiny/#2}}
\begin{frame}[t]
\setlength{\abovedisplayskip}{5pt}
\setlength{\belowdisplayskip}{5pt}
%\frametitle{Semesterplan}
\vspace*{-3pt}
\begin{center}
\hspace*{-4pt}
\begin{tabular}{|r|r|l|}
\hline
\rowcolor{black}
\color{white}\bf Wo&
\color{white}\bf Datum&
\color{white}\bf Inhalt\raisebox{3pt}{\strut}
\\[1pt]
\hline
\rowcolor{darkgreen!10}
\color<1>{darkred} \w{1}{\phantom{0}8}&
\color<1>{darkred} 16.~2.&
\color<1>{darkred}Erweiterungen von $\mathbb{Q}$ um $\sqrt{2}$ und $i$
\raisebox{2pt}{\strut}
\\
\rowcolor{darkgreen!10}
\color<2>{darkred} \w{2}{\phantom{0}9}&
\color<2>{darkred} 23.~2.&
\color<2>{darkred}Holomorphe Funktionen
\\
\rowcolor{darkgreen!10}
\color<3>{darkred} \w{3}{10}&
\color<3>{darkred}  2.~3.&
\color<3>{darkred}Integration entlang eines Pfades
\\
\rowcolor{darkgreen!10}
\color<4>{darkred} \w{4}{11}&
\color<4>{darkred} \phantom{0}9.~3.&
\color<4>{darkred}Cauchy-Integralsatz
\\
\rowcolor{darkgreen!10}
\color<5>{darkred} \w{5}{12}&
\color<5>{darkred} 16.~3.&
\color<5>{darkred}Potenzreihen, analytische Funktionen, meromorphe Funktionen
\\
\rowcolor{darkgreen!10}
\color<6>{darkred} \w{6}{13}&
\color<6>{darkred} 23.~3.&
\color<6>{darkred}Lösung von Differentialgleichungen mit Potenzreihen
\\
\rowcolor{darkgreen!10}
\color<7>{darkred} \w{7}{14}&
\color<7>{darkred} 30.~3.&
\color<7>{darkred}Die Algebra des Integrationsproblems
\\
\rowcolor{gray!40}
\color{white}\w{  }{15}&
\color{white}\phantom{0}6.~4.&
\color{white}keine Vorlesung, Frühjahrsferien
\\
\rowcolor{darkgreen!10}
\color<8>{darkred}\w{8}{16}&
\color<8>{darkred} 13.~4.&
\color<8>{darkred}Die Integration rationaler Funktionen nach Hermite
\\
\rowcolor{darkgreen!10}
\color<9>{darkred}\w{9}{17}&
\color<9>{darkred} 20.~4.&
\color<9>{darkred}Integration in geschlossener Form, 
Prinzip von Liouville
\\
\rowcolor{blue!10}
\color<10>{darkred}\w{10}{18}&
\color<10>{darkred} 27.~4.&
\color<10>{darkred}Vorträge 1: 
\\
\rowcolor{blue!10}
\color<11>{darkred}\w{11}{19}&
\color<11>{darkred} \phantom{0}4.~5.&
\color<11>{darkred}Vorträge 2:
\\
\rowcolor{blue!10}
\color<12>{darkred}\w{12}{20}&
\color<12>{darkred} 11.~5.&
\color<12>{darkred}Vorträge 3:
\\
\rowcolor{blue!10}
\color<13>{darkred}\w{13}{21}&
\color<13>{darkred} 18.~5.&
\color<13>{darkred}Vorträge 4:
\\
\rowcolor{gray!40}
\color{white}\w{  }{22}&
\color{white}25.~5.&
\color{white}keine Sitzung, Pfingstmontag
\\
\rowcolor{blue!10}
\color<14>{darkred}\w{14}{23}&
\color<14>{darkred} \phantom{0}1.~6.&
\color<14>{darkred}Vorträge 5, Abschlusssitzung
\\[1pt]
\hline
\end{tabular}
\end{center}
\end{frame}
\egroup
