%
% 1-potenzreihen.tex -- Potenzreihenentwicklung holomorpher Funktionen
%
% (c) 2025 Prof Dr Andreas Müller
%
\section{Potenzreihenansatz für Differentialgleichungen
\label{buch:differentialgleichungen:section:potenzreihen}}
\kopfrechts{Potenzreihenansatz für Differentialgleichungen}
Wir betrachten eine gewöhnliche Differentialgleichung
\[
F(x,y,y',\dots,y^{(n)}) = 0
\]
der Ordnung $n$.
Wir nehmen an, dass die Lösung sich als Potenzreihe in der
Form
\begin{equation}
y(x)
=
\sum_{k=0}^\infty a_kx^k
\label{buch:differentialgleichungen:potenzreihen:eqn:ansatz}
\end{equation}
schreiben lässt.
Dann kann die Lösung auch als holomorphe Funktion einer komplexen
Variable $x$ betrachtet werden und die Ableitungen von $y(x)$ sind
komplexe Ableitungen.
Für viele Differentialgleichungen wird dann auch die Differentialgleichung
für die holomorphe Funktion erfüllt sein.
Die folgenden Beispiele sollen dies zeigen.

%
% Die Exponentialfunktion
%
\subsection{Die Exponentialfunktion}
Wir betrachten die Differentialgleichung
\begin{equation}
y' = -ky
\label{buch:differentialgleichungen:potenzreihen:eqn:expdgl}
\end{equation}
und setzen den
Potenzreihenansatz~\eqref{buch:differentialgleichungen:potenzreihen:eqn:ansatz}
in die Differentialgleichung ein.
Die Ableitung der  
Potenzreihen~\eqref{buch:differentialgleichungen:potenzreihen:eqn:ansatz}
ist
\[
y'(x)
=
\sum_{k=1}^\infty ka_kx^{k-1}.
\]
Einsetzen in die Differentialgleichung
\eqref{buch:differentialgleichungen:potenzreihen:eqn:expdgl}
ergibt die Gleichung
\begin{align*}
\sum_{l=1}^\infty a_llx^{l-1}
&=
-k
\sum_{l=0}\infty a_lx^l.
\intertext{Um diese besser vergleichen zu können, ersetzen wir $l-1$ durch
$l$, oder $l$ durch $l+1$ auf der linken Seite und erhalten}
\sum_{l=0}^\infty a_{l+1} (l+1)x^{l}
&=
-k
\sum_{l=0}^\infty a_lx^l.
\end{align*}
Durch Vergleich der Koeffizienten der beiden Potenzreihen können wir
schliessen, dass
\begin{equation}
a_{l+1}
=
-k
\frac{1}{l+1} a_l
\label{buch:differentialgleichungen:potenzreihen:eqn:exprek}
\end{equation}
für alle $l\ge 0$.
Wir haben also keine Gleichung, mit der $a_0$ bestimmt werden kann,
aber die Rekursionsgleichung
\eqref{buch:differentialgleichungen:potenzreihen:eqn:exprek}
erlauben uns,
alle Koeffizienten durch $a_0$ auszudrücken:
\begin{align*}
a_1
&=
-k\frac{1}{0+1}a_0 = -ka_0 
\\
a_2
&=
-k\frac{1}{1+1}a_1 = -\frac{k}{2}a_1 = \phantom{-}\frac{k^2}{2}a_0
\\
a_3
&=
-k\frac{1}{2+1}a_2 = -\frac{k}{3}a_2 = -\frac{k^3}{3!}a_0
\\
a_4
&=
-k\frac{1}{3+1}a_3 = -\frac{k}{4}a_3 = \phantom{-}\frac{k^4}{4!}a_0
\\
a_5
&=
-k\frac{1}{4+1}a_4 = -\frac{k}{5}a_4 = -\frac{k^5}{5!}a_0
\\
&\hspace*{1mm}\vdots
\\
a_l
&=
(-1)^l \frac{k^l}{l!}a_0.
\end{align*}
Daraus ergibt sich als Lösung der Differentialgleichung
\[
y(x)
=
\sum_{l=0}^\infty a_lx^l
=
a_0
\sum_{l=0}^\infty
(-1)^l
\frac{k^l}{l!}x^l
=
a_0e^{-kx}.
\]
Tatsächlich ist die Ableitung dieser Funktion
\[
y'(x)
=
-ka_0e^{-kx}
=
-ky(x),
\]
die Exponentialfunktion löst also die Differentialgleichung.

Da 
\eqref{buch:differentialgleichungen:potenzreihen:eqn:expdgl}
eine Differentialgleichung erster Ordnung ist, erwartet man 
genau eine frei wählbare Integrationskonstante.
Der Koeffizient $a_0$ ist diese Konstante.
Sie kann aus einer Anfangsbedingung $y(0)=a_0$ bestimmt werden.

%
% Schwingungen
%
\subsection{Schwingungen}
Die Schwingungsdifferentialgleichung ist eine Differentialgleichung
der Form
\[
y''
=
-\omega^2\,y.
\]
Auch in diesem Fall sind die Koeffizienten konstant und somit
analytisch und es ist gerechtfertigt, eine analytische Lösung
zu erwarten und mit einem Potenzreihenansatz zu suchen.
Die zweite Ableitung des
Ansatzes~\eqref{buch:differentialgleichungen:potenzreihen:eqn:ansatz}
ist
\[
y''(x)
=
\sum_{l=2}^\infty
a_l l(l-1) x^{l-2}
.
\]
Setzen wir dies in die Differentialgleichung ein, ergibt sich
\begin{align*}
\sum_{l=2}^\infty
a_l l(l-1) x^{l-2}
&=
-\omega^2
\sum_{l=0}^\infty
a_lx^l
\intertext{Auch in diesem Fall ist es notwendig, auf der rechten
Seite $l+2$ durch $l$ zu ersetzen, was auf die Gleichung}
\sum_{l=0}^\infty
a_{l+2} (l+1)(l+2) x^{l}
&=
-\omega^2
\sum_{l=0}^\infty
a_lx^l
\end{align*}
Koeffizientenvergleich ergibt die Rekursionsformel
\begin{align*}
(l+1)(l+2)
a_{l+2} 
&=
-\omega^2
a_l
\intertext{für $l>0$ oder}
a_{l+2}
&=
-\frac{\omega^2}{(l+1)(l+2)} a_{l}.
\end{align*}
Diese Rekursionsformeln ermöglichen, die Koeffizienten $a_l$ mit $l\ge 2$
durch $a_0$ und $a_1$ als
\begin{align*}
a_2
&=
-\omega^2 \frac{1}{1\cdot 2} a_0
\\
a_3
&=
-\omega^2 \frac{1}{2\cdot 3} a_1
%=
%-\omega^2 \frac{1}{3!} a_1
\\
a_4
&=
-\omega^2 \frac{1}{3\cdot 4} a_2
=
\phantom{-}\omega^4 \frac{1}{4!} a_0
\\
a_5
&=
-\omega^2 \frac{1}{4\cdot 5} a_3
=
\phantom{-}\omega^4 \frac{1}{5!} a_1
\\
a_6
&=
-\omega^2 \frac{1}{5\cdot 6} a_4
=
-\omega^6 \frac{1}{6!} a_0
\\
a_7
&=
-\omega^2 \frac{1}{6\cdot 7} a_5
=
-\omega^6 \frac{1}{7!} a_1
\intertext{auszudrücken oder allgemein}
a_{2n}   &= (-1)^n \omega^{2n} \frac{1}{(2n)!}   a_0
\\
a_{2n+1} &= (-1)^n \omega^{2n} \frac{1}{(2n+1)!} a_1.
\end{align*}
Daraus kann die Lösung als
\begin{align*}
y(x)
&=
a_0
\sum_{n=0}^\infty
(-1)^n\omega^{2n}\frac{1}{(2n)!} x^{2n}
+
a_1
\sum_{n=0}^\infty
(-1)^n\omega^{2n}\frac{1}{(2n+1)!} x^{2n+1}
\\
&=
a_0
\sum_{n=0}^\infty
\frac{(-1)^n}{(2n)!} (\omega x)^{2n}
+
\frac{a_1}{\omega}
\sum_{n=0}^\infty
\frac{(-1)^n}{(2n+1)!} (\omega x)^{2n+1}
\\
&=
a_0\cos(\omega x)
+
\frac{a_1}{\omega}\sin(\omega x)
\end{align*}
ableiten.
Die Lösungen sind also Linearkombinationen von Sinus-
bzw.~Kosinus-Schwin\-gun\-gen.



