%
% rekursion.tex -- Rekursionsbeziehung zwischen den Koeffizienten
%
% (c) 2021 Prof Dr Andreas Müller, OST Ostschweizer Fachhochschule
%
\documentclass[tikz]{standalone}
\usepackage{amsmath}
\usepackage{times}
\usepackage{txfonts}
\usepackage{pgfplots}
\usepackage{csvsimple}
\usetikzlibrary{arrows,intersections,math}
\definecolor{darkred}{rgb}{0.8,0,0}
\begin{document}
\def\skala{1}
\def\xmax{3}
\def\ymax{3}
\def\h{2.0}
\def\punkt#1#2{
	({(#1)*\h},{(#2)*\h})
}
\begin{tikzpicture}[>=latex,thick,scale=\skala]

\foreach \x in {0,...,\xmax}{
	\draw[color=gray!20,line width=2mm]
		\punkt{\x}{-0.5} -- \punkt{\x}{\ymax+0.5};
}
\foreach \y in {0,...,\ymax}{
	\draw[color=gray!20,line width=2mm]
		\punkt{-0.5}{\y} -- \punkt{\xmax+0.5}{\y};
}
\foreach \x in {0,...,\xmax}{
	\foreach \y in {0,...,\ymax}{
		\fill[color=white] \punkt{\x}{\y} circle[radius=0.1];
		\draw \punkt{\x}{\y} circle[radius=0.1];
	}
}

\draw[->] \punkt{-0.5}{-0.5} -- \punkt{\xmax+0.5}{-0.5}
	coordinate[label={$k$}];
\foreach \x in {0,...,\xmax}{
	\draw \punkt{\x}{-0.5-0.05/\h} -- ++(0,0.1);
}
\draw[->] \punkt{-0.5}{-0.5} -- \punkt{-0.5}{\ymax+0.5}
	coordinate[label={right:$l$}];;
\foreach \y in {0,...,\ymax}{
	\draw \punkt{-0.5-0.05/\h}{\y} -- ++(0.1,0);
}

\node at \punkt{3}{1} [above right] {$a_{k,n-k}x^ky^{n-k}$};
\node at \punkt{1}{3} [above right] {$a_{k,l-2}x^ky^{n-k+2}$};

\draw[->,color=darkred,line width=3pt,shorten <= 1.5mm, shorten >= 1.0mm]
	\punkt{3}{1} -- \punkt{1}{1};
\draw[->,color=blue,line width=3pt,shorten <= 1.5mm, shorten >= 1.0mm]
	\punkt{1}{3} -- \punkt{1}{1};

\node[color=darkred] at \punkt{1.2}{1} [below right]
	{$k(k-1)a_{k,n-k}x^{k-2}y^{n-k}$};
\fill[color=white] \punkt{-0.6}{1.2} rectangle \punkt{-0.4}{1.55};
\node[color=blue] at \punkt{1}{1.2} [above left]
	{$(n-k+2)(n-k+1)a_{k-2,l+2}x^{k-2}y^{n-k}$};

\end{tikzpicture}
\end{document}

