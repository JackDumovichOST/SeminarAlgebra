%
% polynome.tex -- Koeffizienten eines Polynoms in x und y
%
% (c) 2021 Prof Dr Andreas Müller, OST Ostschweizer Fachhochschule
%
\documentclass[tikz]{standalone}
\usepackage{amsmath}
\usepackage{times}
\usepackage{txfonts}
\usepackage{pgfplots}
\usepackage{csvsimple}
\usetikzlibrary{arrows,intersections,math}
\definecolor{darkred}{rgb}{0.8,0,0}
\begin{document}
\def\skala{1}
\def\xmax{5}
\def\ymax{4}
\def\h{1.9}
\def\punkt#1#2{
	({(#1)*\h},{(#2)*\h})
}
\begin{tikzpicture}[>=latex,thick,scale=\skala]

\foreach \x in {0,...,\xmax}{
	\draw[color=gray!20,line width=2mm]
		\punkt{\x}{-0.5} -- \punkt{\x}{\ymax+0.5};
}
\foreach \y in {0,...,\ymax}{
	\draw[color=gray!20,line width=2mm]
		\punkt{-0.5}{\y} -- \punkt{\xmax+0.5}{\y};
}
\foreach \x in {0,...,\xmax}{
	\foreach \y in {0,...,\ymax}{
		\fill[color=white] \punkt{\x}{\y} circle[radius=0.1];
		\draw \punkt{\x}{\y} circle[radius=0.1];
	}
}

\draw[->] \punkt{-0.5}{-0.5} -- \punkt{\xmax+0.5}{-0.5}
	coordinate[label={$k$}];
\foreach \x in {0,...,\xmax}{
	\draw \punkt{\x}{-0.5-0.05/\h} -- ++(0,0.1);
	\node at \punkt{\x}{-0.5} [below] {$\x$};
}
\draw[->] \punkt{-0.5}{-0.5} -- \punkt{-0.5}{\ymax+0.5}
	coordinate[label={right:$l$}];;
\foreach \y in {0,...,\ymax}{
	\draw \punkt{-0.5-0.05/\h}{\y} -- ++(0.1,0);
	\node at \punkt{-0.5}{\y} [left] {$\y$};
}

\node at \punkt{3}{3} [above right] {$a_{kl}x^ky^l$};
\node at \punkt{1}{3} [above right] {$a_{k-2,l}x^{k-2}y^l$};
\node at \punkt{3}{1} [above right] {$a_{k,l-2}x^ky^{l-2}$};
\draw[->,color=darkred,line width=3pt,shorten <= 1.5mm, shorten >= 1.0mm]
	\punkt{3}{3} -- \punkt{1}{3};
\draw[->,color=blue,line width=3pt,shorten <= 1.5mm, shorten >= 1.0mm]
	\punkt{3}{3} -- \punkt{3}{1};

\node[color=darkred] at \punkt{2}{3} [above right]
	{$\displaystyle\frac{\partial^2}{\partial x^2}$};
\node[color=blue] at \punkt{3}{2} [above right]
	{$\displaystyle\frac{\partial^2}{\partial y^2}$};

\end{tikzpicture}
\end{document}

