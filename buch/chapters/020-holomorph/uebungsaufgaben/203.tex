Von einer holomorphen Funktion $f(z)$ ist bekannt, dass 
$\operatorname{Re}f(\bar{z})=\operatorname{Re}f(z)$, $f(z)$ hat
eine konvergente Potenzreihenentwicklung und
$\operatorname{Re} f(x) = p(x)=a_0+a_1x+\dots+a_{n-1}x^{n-1}+a_nx^n$.
Zeigen Sie, dass $f(z) = p(z)=a_0+a_1z+\dots+a_{n-1}z^{n-1}+a_nz^n$.

\begin{loesung}
Der Beweis von Satz~\ref{buch:holomorph:holomorph:satz:polynom}
funktioniert auch für eine Potenzreihe.
Wir schreiben den Realteil von $f(z)$ als
\begin{align*}
\operatorname{Re}f(x+iy)
=
u(x,y)
&=
\sum_{k,l}a_{kl}x^ky^l
\end{align*}
schreiben.
Die Symmetriebedingung
$
\operatorname{Re}f(z)
=
\operatorname{Re}f(\bar{z})
$
bedeutet, dass die Ersetzung $y\mapsto -y$ den Ausdruck für $u(x,y)$
nicht ändert:
\begin{align*}
u(x,y)&=u(x,-y)
&&\Rightarrow&
\sum_{k,l}a_{kl}x^ky^l
&=
\sum_{k,l}a_{kl}x^k(-y)^l
=
\sum_{k,l}(-1)^la_{kl}x^ky^l.
\end{align*}
Aus einem Koeffizientenvergleich schliesst man, dass
\[
a_{kl}=(-1)^l a_{kl}
\quad
\forall k,l
\]
gilt, was $a_{kl}=0$ für $l$ ungerade zur Folge hat.
Aus dem Beweis von Satz~\ref{buch:holomorph:holomorph:satz:polynom}
ist bekannt, dass Koeffizienten $a_{kl}$ mit $l>0$ vollständig durch
die Koeffizienten $a_{k0}$ bestimmt sind.
Die Koeffizienten des Polynoms
$p(z) = a_0+a_1z+a_2z^2+\dots a_nz^n$ müssen
reell sein und es erfüllt
\begin{align*}
\operatorname{Re} f(\bar{z})
&=
\operatorname{Re}
\sum_{k=0}^n
a_k\bar{z}^k
=
\operatorname{Re}
\overline{
\sum_{k=0}^n
a_kz^k
}
=
\operatorname{Re}
\sum_{k=0}^n
a_kz^k
=
\operatorname{Re}
f(z).
\end{align*}
Das Polynom $p(z)$ ist eine Potenzreihenentwicklung, die entlang
der reellen Achse mit $f(z)$ übereinstimmt.
Da die Taylor-Reihe durch die reellen Ableitungen bereits eindeutig
bestimmt ist, muss $f(z)=p(z)$ sein.
\end{loesung}

