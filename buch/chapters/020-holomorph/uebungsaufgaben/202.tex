Betrachten Sie die Funktion $f(z)=z\bar{z}=x^2+y^2$ und berechnen
Sie die Wirtinger-Ableitungen
\[
\frac{\partial f}{\partial z}
\qquad\text{und}\qquad
\frac{\partial f}{\partial \bar{z}}.
\]


\begin{loesung}
Die Definition der Wirtinger-Operatoren ergibt
\begin{align*}
\frac{\partial f}{\partial z}
&=
\frac12\biggl(
\frac{\partial}{\partial x}
-i
\frac{\partial}{\partial y}
\biggr)
(x^2+y^2)
\\
&=
\frac12\bigl(
2x-i\cdot 2y
\bigr)
=
\bar{z}
\\
\frac{\partial f}{\partial \bar{z}}
&=
\frac12\biggl(
\frac{\partial}{\partial x}
+i
\frac{\partial}{\partial y}
\biggr)
(x^2+y^2)
\\
&=
\frac12\bigl(
2x+i\cdot 2y
\bigr)
=
z.
\end{align*}
Formal entspricht dies genau der Produktregel
\[
\frac{\partial}{\partial z}(z\bar{z})
=
\bar{z}
\qquad
\text{und}
\qquad
\frac{\partial}{\partial \bar{z}}(z\bar{z})
=
z
\]
für das Produkt $z\bar{z}$,
indem $z$ und $\bar{z}$ als unabhängige Variablen betrachtet
werden können.
\end{loesung}
