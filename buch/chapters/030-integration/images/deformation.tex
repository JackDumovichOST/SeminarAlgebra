%
% deformation.tex -- Deformation eines zusammenziehbaren Weges
%
% (c) 2021 Prof Dr Andreas Müller, OST Ostschweizer Fachhochschule
%
\documentclass[tikz]{standalone}
\usepackage{amsmath}
\usepackage{times}
\usepackage{txfonts}
\usepackage{pgfplots}
\usepackage{csvsimple}
\usetikzlibrary{arrows,intersections,math,calc}
\begin{document}
\def\skala{1}
\def\a{5}
\def\b{3}
\def\c{-4}
\definecolor{darkred}{rgb}{0.8,0,0}
\definecolor{darkgreen}{rgb}{0,0.6,0}
\begin{tikzpicture}[>=latex,thick,scale=\skala,
declare function = {
aX(\u) = \a * (cos(180 - 60 * \u) + 1);
aY(\u) = \a * sin(180 - 60 * \u);
tX(\u) =  sin(180 - 60 * \u);
tY(\u) = -cos(180 - 60 * \u);
nX(\u) =  cos(180 - 60 * \u);
nY(\u) =  sin(180 - 60 * \u);
Y(\u,\v) = aX(\v) + \b * (2 * \u - 1) * nX(\v) + \c * (0.5 - \u) * \u * \u * tX(\v);
X(\u,\v) = aY(\v) + \b * (2 * \u - 1) * nY(\v) + \c * (0.5 - \u) * \u * \u * tY(\v);
uu(\t,\s) = (\s * sin(360 * \t) + 1) / 2;
vv(\t,\s) = \s * (1 - cos(360 * \t)) / 2;
XX(\t,\s) = X(uu(\t,\s),vv(\t,\s));
YY(\t,\s) = Y(uu(\t,\s),vv(\t,\s));
}]

%\foreach \s in {0,0.1,...,1.001}{
%	\draw plot[domain=0:1] ({X(\x,\s)},{Y(\x,\s)});
%}
%
%\foreach \t in {0,0.1,...,1.001}{
%	\draw plot[domain=0:1] ({X(\t,\x)},{Y(\t,\x)});
%}

\foreach \s in {0.1,0.2,...,1}{
	\draw[color=darkred!20,line width=0.7pt]
		plot[domain=0:1,samples=100]
			({XX(\x,\s)},{YY(\x,\s)});
}

\def\s{0.75}
\def\tminus{0.498}
\draw[color=blue,line width=1.2pt]
	plot[domain=0:\tminus,samples=100] 
		({XX(\x,\s)},{YY(\x,\s)})
	--
	plot[domain=\s:1]
		({XX(\tminus,\x)},{YY(\tminus,\x)});

\def\s{0.44}
\def\tplus{0.502}
\draw[color=blue,line width=1.2pt]
	plot[domain=1:\s]
		({XX(\tplus,\x)},{YY(\tplus,\x)})
	--
	plot[domain=\tplus:1,samples=100] 
		({XX(\x,\s)},{YY(\x,\s)});

\begin{scope}[xshift=-0.7cm,yshift=-2.5cm]
	\draw[->] (-0.05,0) -- (7.3,0)
		coordinate[label={$\operatorname{Re}$}];
	\draw[->] (0,-0.05) -- (0,6.3)
		coordinate[label={right:$\operatorname{Im}$}];
\end{scope}

\draw[color=darkgreen,line width=1.2pt]
	plot[domain=0:1,samples=100]
		({XX(0.5,\x)},{YY(0.5,\x)});

% Kurve \gamma
\draw[color=darkred,line width=1.2pt]
	plot[domain=0:1,samples=200]
		({XX(\x,1)},{YY(\x,1)});
\node[color=darkred] at (3,3.4) {$\gamma(t) = H(t,1)$};

% Punkt a
\coordinate (A) at (0,0);
\fill[color=darkred] (A) circle[radius=0.05];
\node[color=darkred] at (A) [left] {$a$};

% Punkt b
\coordinate (B) at (4.33,2.5);
\fill[color=darkred] (B) circle[radius=0.05];
\node[color=darkred] at ($(B)+(0,-0.2)$) [above right] {$b=H(\frac12,1)$};

\begin{scope}[xshift=2.0cm,yshift=0.1cm]
	\node[color=darkgreen] (0,1) [rotate=20] {$s\mapsto H(\frac12,s)$};
\end{scope}

\begin{scope}[xshift=3.0cm,yshift=1.4cm]
	\node[color=blue] (0,1) [rotate=40] {$s\mapsto H(t,s)$};
\end{scope}

\node[color=blue] at (1.2,1.6) {$t\mapsto H(t,s)$};

\node[color=darkred] at (5,-1.8) {$\gamma_1$};
\node[color=darkred] at (0.5,3) {$\gamma_2$};

\end{tikzpicture}
\end{document}

