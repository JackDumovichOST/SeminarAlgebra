%
% beliebig.tex -- Zusammensetzung eines beliebigen Weges
%
% (c) 2021 Prof Dr Andreas Müller, OST Ostschweizer Fachhochschule
%
\documentclass[tikz]{standalone}
\usepackage{amsmath}
\usepackage{times}
\usepackage{txfonts}
\usepackage{pgfplots}
\usepackage{csvsimple}
\usetikzlibrary{arrows,intersections,math,calc}
\definecolor{darkred}{rgb}{0.8,0,0}
\definecolor{darkgreen}{rgb}{0,0.6,0}
\begin{document}
\def\skala{1}
\begin{tikzpicture}[>=latex,thick,scale=\skala]

\clip (0.5,0.5) rectangle (11.5,9.5);

\foreach \x in {0,...,12}{
	\draw[color=gray] (\x,0) -- ++(0,10);
}
\foreach \y in {0,...,10}{
	\draw[color=gray] (0,\y) -- ++(12,0);
}

\coordinate (A1) at (4,1);
\coordinate (A2) at (6,2);
\coordinate (A3) at (8,3);
\coordinate (A4) at (11,7);
\coordinate (A5) at (9,9);
\coordinate (A6) at (4,8);
\coordinate (A7) at (3,9);
\coordinate (A8) at (1,7);
\coordinate (A9) at (3,6);
\coordinate (A10) at (4,5);
\coordinate (A11) at (2,4);
\coordinate (A12) at (1,3);

%\node at (A1) {1};
%\node at (A2) {2};
%\node at (A3) {3};
%\node at (A4) {4};
%\node at (A5) {5};
%\node at (A6) {6};
%\node at (A7) {7};
%\node at (A8) {8};
%\node at (A9) {9};
%\node at (A10) {10};
%\node at (A11) {11};
%\node at (A12) {12};


\begin{scope}
	\clip (A1) rectangle (A2);
	\fill[color=darkred!20,opacity=0.5]
		(A1) to[out=0,in=-135] (A2) -- (4,2) -- cycle;
	\draw[color=blue,line width=2pt] (A1) -- (4,2) -- (A2);
\end{scope}

\begin{scope}
	\clip (A2) rectangle (A3);
	\fill[color=darkred!20,opacity=0.5]
		(A2) to[out=45,in=-120] (A3) -- (6,3) -- cycle;
	\draw[color=blue,line width=2pt] (A2) -- (6,3) -- (A3);
\end{scope}

\begin{scope}
	\clip (A3) rectangle (A4);
	\fill[color=darkred!20,opacity=0.5]
		(A3) to[out=60,in=-90] (A4) -- (8,7) -- cycle;
	\draw[color=blue,line width=2pt] (A3) -- (8,7) -- (A4);
\end{scope}

\begin{scope}
	\clip (A4) rectangle (A5);
	\fill[color=darkred!20,opacity=0.5]
		(A4) to[out=90,in=0] (A5) -- (9,7) -- cycle;
	\draw[color=blue,line width=2pt] (A4) -- (9,7) -- (A5);
\end{scope}

\begin{scope}
	\clip (A5) rectangle (A6);
	\fill[color=darkred!20,opacity=0.5]
		(A5) to[out=180,in=0] (A6) -- (9,8) -- cycle;
	\draw[color=blue,line width=2pt] (A5) -- (9,3) -- (A6);
\end{scope}

\begin{scope}
	\clip (A6) rectangle (A7);
	\fill[color=darkred!20,opacity=0.5]
		(A6) to[out=180,in=0] (A7) -- (3,8) -- cycle;
	\draw[color=blue,line width=2pt] (A6) -- (3,8) -- (A7);
\end{scope}

\begin{scope}
	\clip (A7) rectangle (A8);
	\fill[color=darkred!20,opacity=0.5]
		(A7) to[out=180,in=90] (A8) -- (3,7) -- cycle;
	\draw[color=blue,line width=2pt] (A7) -- (3,7) -- (A8);
\end{scope}

\begin{scope}
	\clip (A8) rectangle (A9);
	\fill[color=darkred!20,opacity=0.5]
		(A8) to[out=-90,in=180] (A9) -- (3,7) -- cycle;
	\draw[color=blue,line width=2pt] (A8) -- (3,7) -- (A9);
\end{scope}

\begin{scope}
	\clip (A9) rectangle (A10);
	\fill[color=darkred!20,opacity=0.5]
		(A9) to[out=0,in=90] (A10) -- (4,6) -- cycle;
	\draw[color=blue,line width=2pt] (A9) -- (4,6) -- (A10);
\end{scope}

\begin{scope}
	\clip (A10) rectangle (A11);
	\fill[color=darkred!20,opacity=0.5]
		(A10) to[out=-90,in=30] (A11) -- (4,4) -- cycle;
	\draw[color=blue,line width=2pt] (A10) -- (4,4) -- (A11);
\end{scope}

\begin{scope}
	\clip (A11) rectangle (A12);
	\fill[color=darkred!20,opacity=0.5]
		(A11) to[out=-150,in=90] (A12) -- (2,3) -- cycle;
	\draw[color=blue,line width=2pt] (A11) -- (2,3) -- (A12);
\end{scope}

\begin{scope}
	\clip (A12) rectangle (A1);
	\fill[color=darkred!20,opacity=0.5]
		(A12) to[out=-90,in=180] (A1) -- (4,3) -- cycle;
	\draw[color=blue,line width=2pt] (A12) -- (4,3) -- (A1);
\end{scope}

\def\rechteck#1#2#3{
\begin{scope}
	\clip #1 rectangle #2;
	\fill[color=darkgreen!20,opacity=0.5] #1 rectangle #2;
	\draw[color=darkgreen,line width=2pt] #1 rectangle #2;
	\node[color=darkgreen] at ($0.5*#1+0.5*#2$) {$#3$};
\end{scope}
}

\rechteck{(4,2)}{(6,7)}{R_1}
\rechteck{(6,3)}{(8,7)}{R_2}
\rechteck{(2,3)}{(4,4)}{R_3}
\rechteck{(3,6)}{(4,8)}{R_4}
\rechteck{(4,7)}{(9,8)}{R_5}

\draw[color=darkred,line width=2.0pt]
	(A1) to[out=0,in=-135]
	(A2) to[out=45,in=-120]
	(A3) to[out=60,in=-90]
	(A4) to[out=90,in=0]
	(A5) to[out=180,in=0]
	(A6) to[out=180,in=0]
	(A7) to[out=180,in=90]
	(A8) to[out=-90,in=180]
	(A9) to[out=0,in=90]
	(A10) to[out=-90,in=30]
	(A11) to[out=-150,in=90]
	(A12) to[out=-90,in=180]
	cycle
;

\node[color=darkred] at (2.5,2.1) {$H_{1}$};
\node[color=darkred] at (4.5,1.6) {$H_{2}$};
\node[color=darkred] at (6.5,2.6) {$H_{3}$};
\node[color=darkred] at (9,6) {$H_{4}$};
\node[color=darkred] at (9.7,7.7) {$H_{5}$};
\node[color=darkred] at (8,8.4) {$H_{6}$};
\node[color=darkred] at (3.3,8.4) {$H_{7}$};
\node[color=darkred] at (2.3,7.7) {$H_{8}$};
\node[color=darkred] at (2.3,6.5) {$H_{9}$};
\node[color=darkred] at (3.7,5.8) {$H_{10}$};
\node[color=darkred] at (3.7,4.2) {$H_{11}$};
\node[color=darkred] at (1.6,3.3) {$H_{12}$};

\end{tikzpicture}
\end{document}

