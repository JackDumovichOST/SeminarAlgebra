%
% fig-beliebig.tex
%
% (c) 2026 Prof Dr Andreas Müller
%
\begin{figure}
\centering
\includegraphics{chapters/030-integration/images/beliebig.pdf}
\caption{Ein Weg $\gamma$ berandet ein Gebiet, welches sich aus Teilgebieten
zusammensetzen lässt, welche entweder Rechtecke oder Dreiecke sind.
Alle Teilgebiete werden so orientiert, dass der Rand das Teilgebiet
im Gegenuhrzeigersinn umläuft.
Dann werden innere Teilgebietesgrenzen jeweils in entgegengesetzter
Richtung durchlaufen.
In der Summe der Randintegrale heben sie sich weg, so dass das
Wegintegral über $\gamma$ übrig bleibt.
\label{buch:integration:wegintegral:fig:beliebig}}
\end{figure}
