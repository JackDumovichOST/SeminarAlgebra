%
% 2-cauchy.tex -- 2. Cauchy-Integralsatz
%
% (c) 2025 Prof Dr Andreas Müller
%
\section{Der Residuensatz
\label{buch:integration:section:cauchy}}
\kopfrechts{Der Integralsatz von Cauchy}
Das Beispiel~\ref{buch:integration:wege:bsp:kreisintegral} hat
gezeigt, dass mit Ausnahme des Falls $n=-1$ die Potenzen $z^n$ 
das Wegintegral über einen Kreis um den Nullpunkt den Wert 0
ergab.
Für $\frac{1}{z}$ ergab sich der Wert
\[
\oint_{S^1} \frac{dz}{z}
=
2\pi i.
\]
Der Homotopiebegriff erlaubt, die Wege beliebig zu deformieren,
solange der Definitionsbereich der Funktion nicht verlassen wird.
Es scheint, dass das Integral über geschlossene Wege vor allem davon
abhängt, was in Polstellen des Integranden passiert.
Der Residuensatz, der in diesem Abschnitt hergeleitet werden soll,
formalisiert diese Beobachtung.

%
% Wegintegrale über geschlossene Wege
%
\subsection{Wegintegrale über geschlossene Wege}
\input{chapters/030-integration/fig/fig-residuum.tex}%
Nach dem Satz~\ref{buch:integration:wegintegral:satz:zusammenziehbar}
von Abschnitt~\ref{buch:integration:wegintegral:subsection:homotopie}
verschwindet das Integral einer holomorphen Funktion entlang eines
zusammenziehbaren Weges.
Wir betrachten jetzt eine holomorphe Funktion $f$ in einem Gebiet $U$
und einen Punkt $z_0\in U$
und einen Weg $\gamma$, der den Punkt $z_0\in U$
einmal im Gegenuhrzeigersinn umläuft
(siehe Abbildung\ref{buch:integration:fig:residuum}).
Die Funktion
\[
f_0
\colon
U\setminus\{z_0\}
\to
\mathbb{C}
:
z
\mapsto
\frac{f(z)}{z-z_0}
\]
ist immer noch holomorph.
Der Weg $\gamma$ ist jetzt im Definitionsgebiet $U\setminus \{z_0\}$
der Funktion $f_0$ nicht mehr zusammenziehbar, er müsste über den Punkt
$z_0$ hinweggezogen werden, wo die Funktion $f_0$ nicht definiert ist.
Man kann also nicht erwarten, dass das Integral von $f_0(z)$ über
den Weg $\gamma$ verschwindet.
Vielmehr gilt der folgende, viel überraschendere Satz.

\begin{satz}[Cauchy]
\label{buch:integration:satz:residuensatz}
\index{Satz!Cauchy-Integralformel}%
Sei $f\colon U\to\mathbb{C}$ holomorphe Funktion und $\gamma\colon I\to U$
ein geschlossener Weg, der den Punkt $z_0$ einmal im Gegenuhrzeigersinn
umläuft.
Dann gilt
\begin{equation}
f(z_0)
=
\frac{1}{2\pi i}
\oint_{\gamma} \frac{f(z)}{z-z_0}\,dz.
\label{buch:integration:eqn:residuensatz}
\end{equation}
\end{satz}

\begin{proof}
Die Berechnung des Integrals
\eqref{buch:integration:eqn:residuensatz}
wird dadurch erschwert, dass der Weg $\gamma$ den Punkt $z_0$
umläuft, der nicht im Definitionsgebiet liegt.
Wir ergänzen ihn daher zu einem geschlossenen Weg, der ein
ganz in $U\setminus\{z_0\}$ liegendes Gebiet berandet.
Dazu fügen wir beim Punkt $A$ die in 
Abbildung\ref{buch:integration:fig:residuum}
blau eingezeichneten Wegstücke $\gamma_-$, $\gamma_1$ und $\gamma_+$ 
ein.
In der Zeichnung haben die Wege $\gamma_-$ und $\gamma_+$ aus Gründen
der Darstellung einen kleinen Abstand, in Wahrheit sollen diese beiden
Wegstücke bis auf die Durchlaufrichtung identisch sein.
Nach dem Homotpiesatz gilt jetzt
\begin{equation}
\int_{\gamma
\cdot
\gamma_-
\cdot
\gamma_1
\cdot
\gamma_+
}
\frac{f(z)}{z-z_0}\,dz
=
0,
\label{buch:integration:residuensatz:eqn:proof1}
\end{equation}
weil die Funktion im hellrot gefärbten Gebiet, welches vom zusammengesetzten
Weg berandet wird, holomorph ist.

Die Wegstücke $\gamma_-$ und $\gamma_+$ werden in entgegengesetzter 
Richtung durchlaufen.
Verkettet bilden sie einen zusammenziehbaren Weg im Definitionsgebiet
der der Funktion $f_0$.
Die Wegintegrale über diese beiden Teilstücke heben sich also weg.

Vom Integral auf der linken Seite von
\eqref{buch:integration:residuensatz:eqn:proof1}
bleiben jetzt nur noch die Integrale über die beiden Wegstücke
$\gamma$ und $\gamma_1$,
die sich wegen \eqref{buch:integration:residuensatz:eqn:proof1}
ebenfalls wegheben müssen.
Es gilt also
\begin{equation}
0
=
\int_{\gamma+\gamma_1}
\frac{f(z)}{z-z_0}
\,dz
=
\oint_{\gamma}
\frac{f(z)}{z-z_0}
\,dz
+
\oint_{\gamma_1}
\frac{f(z)}{z-z_0}
\,dz
\quad\Rightarrow\quad
\oint_\gamma
\frac{f(z)}{z-z_0}
\,dz
=
-
\oint_{\gamma_1}
\frac{f(z)}{z-z_0}
\,dz.
\label{buch:integration:residuensatz:eqn:proof2}
\end{equation}

Zur Berechnung des Integrals über $\gamma_1$ verwenden wir die
Parametrisierung
\[
\gamma_1
\colon
I
\to
\mathbb{C}
:
t
\mapsto
\gamma_1(t)
=
z_0
+
re^{-2\pi i t}.
\]
mit der Ableitung 
\[
\dot{\gamma}_1(t)
=
-2\pi i r e^{-2\pi it}.
\]
Damit können wir das Wegintegral über $\gamma_1$ berechnen und
erhalten
\begin{align}
\int_{\gamma_1}
\frac{f(z)}{z-z_0}\,dz
&=
\int_0^1
f_0(\gamma_1(t))\cdot \dot{\gamma}_1(t)\,dt
\\
&=
\int_0^1
\frac{f(z_0+re^{-2\pi it})}{e^{-2\pi it}}
\cdot(-2\pi i r e^{-2\pi it})
\,dt
\notag
\\
&=
-2\pi i
\int_0^1
f(z_0+re^{-2\pi it})
\,dt.
\notag
\intertext{Nach dem Homotpiesatz~\ref{buch:integration:satz:homotopie} 
bleibt das letzte Integral gleich, wenn der Radius $r$ des kleinen
Kreises $\gamma_1$ verkleinert wird.
Insbesondere gilt im Grenzübergang $r\to 0$}
\int_{\gamma_1}
\frac{f(z)}{z-z_0}\,dz
&=
\lim_{r\to 0}
\biggl(
-2\pi i
\int_0^1
f(z_0+re^{-2\pi it})
\,dt
\biggr)
\notag
\\
&=
-2\pi i
\int_0^1
f(z_0)\,dt
=
-2\pi if(z_0).
\label{buch:integration:residuensatz:eqn:proof3}
\end{align}
Indem man das Resultat
\eqref{buch:integration:residuensatz:eqn:proof3}
in
\eqref{buch:integration:residuensatz:eqn:proof2}
einsetzt, findet man jetzt
\[
\oint_\gamma
\frac{f(z)}{z-z_0}
\,dz
=
2\pi if(z_0)
\qquad\Rightarrow\qquad
f(z_0)
=
\frac{1}{2\pi i}
\oint_\gamma
\frac{f(z)}{z-z_0}
\,dz.
\]
Damit ist der Satz bewiesen.
\end{proof}

Der Satz sagt, dass die Werte der holomoprhen Funktion $f(z)$ im
Inneren des Weges vollständig durch die Werte der Funktion auf
dem Weg $\gamma$ bestimmt sind.

%
% Wegintegrale und Pole
%
\subsection{Wegintegrale und Pole}
Sei $f\colon U\to\mathbb{C}$ eine holomorphe Funktion und
$a\in U$ ein Punkt.
Sei weiter $\gamma$ ein geschlossener Weg, der den Punkt $a$
einmal im Gegenuhrzeigersinn umkreist.

%
% Mittelwerteigenschaft
%
\subsubsection{Mittelwerteigenschaft}
Aus der Formel~\eqref{XXX} folgt auch sofort die folgende Eigenschaft
einer harmonischen Funktion.

\begin{satz}[Mittelwerteigenschaft]
\label{buch:integration:cauchy:satz:mittelwerteigenschaft}
Sei $U\subset\mathbb{C}$ ein Gebiet und $z_0\in U$ ein Punkt in $U$.
Weiter sei $u$ eine harmonische Funktion in $U$ und $r$ der Radius
eines Kreises um den Punkt $z_0$, der immer noch ganz in $U$ enhalten ist.
Dann ist
\begin{equation}
u(z_0)
=
\frac{1}{2\pi}
\int_0^{2\pi} u(z_0 + re^{it}) \,dt.
\label{buch:integration:cauchy:eqn:mittelwerteigenschaft}
\end{equation}
\end{satz}

\begin{proof}
Zunächst kann nach Satz~\ref{buch:holomorph:holomorph:satz:utov}
eine Funktion $v$ gefunden werden, so dass in einer Umgebung, die auch
den Kreis mit Radius $r$ um $z_0$ enthält, $f(z) = u(z) + iv(z)$ eine
holomorphe Funktion ist.
Nach dem Residuensatz ist dann
\[
f(z_0)
=
\frac{1}{2\pi i}
\oint_{S^1_r} 
\frac{f(z)}{z-z_0}
\,dz
\]
für den Kreis $S^1_r$ mit Radius $r$ um den Punkt $z_0$.
Das Integral kann man durch $t\mapsto z_0+e^{it}$ parametrisieren,
dann wird es zu
\begin{align*}
f(z_0)
&=
\frac{1}{2\pi i}
\int_0^{2\pi}
\frac{ f(z_0+re^{it}) }{z_0+re^{it}-z_0} ir e^{it}\,dt
=
\frac{1}{2\pi}
\int_0^{2\pi}
\frac{ f(z_0+re^{it}) }{re^{it}} r e^{it}\,dt
\\
&=
\frac{1}{2\pi}
\int_0^{2\pi}
 f(z_0+re^{it})\,dt.
\intertext{Den Integranden kann man in Real- und Imaginärteil aufspalten und
erhält}
u(z) + iv(z)
&=
\frac{1}{2\pi}
\int_0^{2\pi}
u(z_0+re^{it})\,dt
+
\frac{i}{2\pi}
\int_0^{2\pi}
v(z_0+re^{it})\,dt.
\end{align*}
Die Formel~\eqref{buch:integration:cauchy:eqn:mittelwerteigenschaft}
gilt daher sowohl für $u$ wie auch für $v$ ganz unabhängig von der
Wahl der Funktion $v$.
\end{proof}

%
% Maximumprinzip
%
\subsubsection{Maximumprinzip}
Aus der Mittelwerteigenschaft folgt eine weitere wichtige Eigenschaft
harmonischer Funktionen, nämlich das Maximum- und Minimum-Prinzip.

\begin{satz}[Maximumprinzip]
Sei $U\subset\mathbb{C}$ ein Gebiet und $u(x,y)$ ein in $U$ harmonische
Funktion.
Dann nimmt die Funktio $u$ ihr Maximum und Minimum auf dem Rand von $U$ an.
\end{satz}

\begin{proof}
Wir zeigen, dass das Maximum oder Minimum nicht im Inneren des Gebietes
$U$ liegen kann.
Dazu nehmen wir an, das Maximum liegt im Punkt $z_0\in U$.
Sei $r>0$ der Radius eines Kreises um den Punkt $z_0$, der vollständig
in $U$ enthalten ist.
Dann ist $u(z_0+ re^{i\varphi}) < u(z_0)$ für alle Winkel $\varphi$,
die Werte auf dem Kreis sind kleiner als der Wert im Punkt $z_0$.
Da die Werte auf dem Kreis kleiner sind als der Wert $u(z_0)$,
ist der Mittelwert $<u(z_0)$.
Nach Satz~\ref{buch:integration:cauchy:satz:mittelwerteigenschaft} ist
der Mittelwert der Werte auf dem Kreis mit Radius $r$ um $z_0$ gleich dem
Wert $u(z_0)$, ein Widerspruch.
Der Widerspruch zeigt, dass $z_0$ nicht ein Maximum sein kann.

Auf die gleiche Weise kann man den Fall eines Minimums behandeln.
\end{proof}

Das Maximumprinzip gilt für die Lösungen einer homogenen partiellen
Differentialgleichung mit einem beliebigen elliptischen Operator.
Der Beweis folgt allerdings einem ganz anderen Weg, denn die
Mittelwerteigenschaft gilt für solche Lösungen nicht.
Für eine ausführliche Diskussion siehe \cite[section 6.4]{buch:evans}.

%
% Ableitungen
%
\subsection{Ableitungen}
Der Satz~\ref{buch:integration:satz:residuensatz} zeigt, wie der
Wert einer holomorphen Funktion in einem Punkt $z_0$ allein aus den
Werten entlang eines Weges berechnet werden kann, der den Punkt $z_0$
im Gegenuhrzeigersinn umläuft.
Die durch die Formel
\eqref{buch:integration:eqn:residuensatz}
gegebene Funktion ist holomorph, man sollte daher auch die Ableitung
mit Hilfe eines Integrals berechnen können.
Wenn man 
\eqref{buch:integration:eqn:residuensatz}
unter dem Integral nach $z_0$ ableitet, erhält man
\begin{align}
f'(z_0)
=
\frac{d}{dz_0}f(z_0)
&=
\frac{1}{2\pi i}
\oint
\frac{d}{dz_0}
\frac{f(z)}{z-z_0}
\,dz
\notag
\\
&=
\frac{1}{2\pi i}
\oint
\frac{f(z)}{(z-z_0)^2}
\,dz.
\label{buch:integration:eqn:ableitung2}
\end{align}
Wie erwartet kann die Ableitung von $f$ ebenfalls durch ein einfaches
Wegintegral berechnet werden.
Durch wiederholtes Ableiten kann man auch höhere Ableitungen finden,
wie sie der folgende Satz zeigt.

\begin{satz}[Cauchy-Formel für Ableitungen]
\label{buch:integration:satz:ableitungn}
Ist $f\colon U\to\mathbb{C}$ eine holomorphe Funktion und $\gamma$
ein Weg, der den Punkt $z_0\in U$ im Gegenuhrzeigersinn umläuft.
Dann gilt
\begin{equation}
\frac{d^n}{dz_0^n}f(z_0)
=
f^{(n)}(z_0)
=
\frac{n!}{2\pi i}
\oint
\frac{f(z)}{(z-z_0)^{n+1}}
\,dz.
\label{buch:integration:eqn:ableitungn}
\end{equation}
\end{satz}

\begin{proof}
Der Fall $n=1$ ist bereits durch \eqref{buch:integration:eqn:ableitung2}
bewiesen, wir müssen die Formel
\eqref{buch:integration:eqn:ableitungn}
für $n>1$ beweisen.
Wir tun dies mit Hilfe vollständiger Induktion, wobei 
\eqref{buch:integration:eqn:ableitung2} als Induktionsverankerung
dient.

Wir nehmen jetzt an \eqref{buch:integration:eqn:ableitungn} für $n$ gilt,
und zeigen, dass sie dann auch für $n+1$ gilt.
Dazu leiten wir \eqref{buch:integration:eqn:ableitungn} nach $z_0$
ab und erhalten
\begin{align*}
f^{(n+1)}(z_0)
&=
\frac{d}{dz_0}
f^{(n)}(z_0)
=
\frac{d}{dz_0}
\frac{n!}{2\pi i}
\oint
\frac{f(z)}{(z-z_0)^{n+1}}
\,dz
\\
&=
\frac{n!}{2\pi i}
\oint
\frac{d}{dz_0}
\frac{f(z)}{(z-z_0)^{n+1}}
\,dz
\\
&=
\frac{n!}{2\pi i}
\oint
(-(n+1))
\frac{f(z)}{(z-z_0)^{n+2}}
(-1)
\,dz
\\
&=
\frac{(n+1)!}{2\pi i}
\oint
\frac{f(z)}{(z-z_0)^{n+2}}
\,dz.
\end{align*}
Damit ist \eqref{buch:integration:eqn:ableitungn} bewiesen.
\end{proof}

%
% Der Satz von Liouville
%
\subsection{Der Satz von Liouville}
Die Cauchy-Formel für die Ableitungen hat die unerwartete Konsquenz,
dass auf ganz $\mathbb{C}$ definierte holomorphe Funktionen, die nicht
Polynome sind, exponentiell schnell anwachsen müssen, wenn $z$ gross wird.

\begin{definition}[polynomiell beschränkt]
Eine in ganz $\mathbb{C}$ definierte komplexe Funktion $f(z)$ heisst
\emph{polynomiell beschränkt}, wenn es $N\in\mathbb{N}$ und $a>0$
\index{polynomiell beschränkt}%
gibt derart, dass
\begin{equation}
|f(z)|
\le a(1+|z|^N)
\label{buch:integration:eqn:liouvilleschranke}
\end{equation}
für alle $z\in\mathbb{C}$ gilt.
\end{definition}

\begin{satz}[Liouville]
\label{buch:integration:satz:liouville}
\index{Satz!von Liouville}%
Sei $f$ eine auf ganz $\mathbb{C}$ holomorphe Funktion, die polynomiell
beschränkt ist.
Dann ist $f$ ein Polynom vom Grad höchstens $N$.
\end{satz}

\begin{proof}
Wir wenden den Satz
\ref{buch:integration:satz:ableitungn}
für Ableitungen einer holomorphen Funktion auf den Weg
$\gamma(t)=z_0+Re^{2\pi it}$.
Nach der Formel \eqref{buch:integration:eqn:ableitungn}
gilt für für alle $k$
\begin{align*}
f^{(k)}(z_0)
&=
\frac{k!}{2\pi i}
\oint_\gamma
\frac{f(z)}{(z-z_0)^{k+1}}
\,dz.
\intertext{Mit der Voraussetzung
\eqref{buch:integration:eqn:liouvilleschranke}
wir dies durch}
|f^{(k)}(z_0)|
&=
\frac{k!}{2\pi}
\oint_\gamma
\biggl|
\frac{f(z)}{(z-z_0)^{k+1}}
\biggr|
\,dz
\\
&\le
\frac{k!}{2\pi}
\int_0^1
\frac{|f(\gamma(t))|}{R^{k+1}}
2\pi R
\,dt
\\
&\le
k!
\int_0^1
\frac{|f(\gamma(t))|}{R^k}
\,dt
\\
&\le
\frac{k!}{R^k}
\int_0^1
a(1+|\gamma(t)|^N)
\,dt
\\
&\le
\frac{k!}{R^k}
\int_0^1
a(1+(|z_0|+R)^N)
\,dt
\\
&\le
k!
a
\frac{
1+(|z_0|+R)^N
}{R^k}
\end{align*}
abgeschätzt.
Für $R\to\infty$ strebt der Bruch auf der rechten Seite geht für $k>N$ 
gegen $0$, so dass folgt $f^{(k)}(z_0)=0$.
Alle Ableitungen höherer als $N$-ter Ordnung der Funktion $f$ verschwinden,
also ist $f$ ein Polynom vom Grad höchstens $N$.
\end{proof}

Man beachte, dass der Grenzübergang im letzten Schritt des Beweises 
verwendet, dass die Funktion auf ganz $\mathbb{C}$ holomorph ist.
Sobald es Punkte in $\mathbb{C}$ gibt, in denen die Funktion nicht
holomorph ist, gilt auch die Integralformel nicht mehr und damit
bricht die Abschätzungskette des Beweises zusammen.

Der Spezialfall $N=0$ liefert die folgende Aussage.

\begin{korollar}
Eine beschränkte, auf ganz $\mathbb{C}$ holomorphe Funktion ist
konstant.
\end{korollar}

Für reelle Funktionen kann ein solches Resultat nicht gelten.
Die trigonometrischen Funktionen $\sin x$ und $\cos x$ sind
für alle reellen Argumente von $x$ beschränkt.
In $\mathbb{R}$ gilt für beide Funktionen
\[
|f(x)| \le 1,
\]
was der Ungleichung
\eqref{buch:integration:eqn:liouvilleschranke}
mit $N=0$ und $a=1$ entspricht.
Die trigonometrischen Funktionen sind aber alles andere als konstant
und auch keine Polynome, da Ableitungen beliebig hoher Ordnung nicht
verschwinden.

Umgekehrt bedeutet der Satz von Liouville, dass eine Funktion,
die durch eine für alle $z\in\mathbb{C}$ konvergente Potenzreihe
definiert ist, die kein Polynom ist, nicht polynomiell beschränkt
sein kann.
Für die Sinus- und Kosinusfunktionen zeigt die Definition
\[
\cos z = \frac{e^{iz}+e^{-iz}}{2}
\qquad\text{und}\qquad
\sin z = \frac{e^{z}-e^{-iz}}{2i},
\]
dass sich die Funktionen für imaginäre Argumente wie Exponentialfunktionen
verhalten, die nicht polynomiell beschränkt sind.

Holomorphe Funktionen, die keine Polynome sind, aber für grosse $z$
nur polynomiell anwachsen, können nicht in ganz $\mathbb{C}$
holomorph sein.
Rationale Funktionen, bei denen der Nennergrad grösser ist als der
Zählergrad, sind Beispiele für dieses Phänomen.
Tatsächlich hat eine solche Funktion immer mindestens eine Polstelle,
in der die Funktion nicht definiert ist.

%
% Die Umlaufzahl
%
\subsection{Die Umlaufzahl}
Die Sätze der vorangegangenen Abschnitte mussten jeweils die etwas
schwerfällige Voraussetzung über den Weg $\gamma$ machen, dass er
den Punkt $z_0$ im Gegenuhrzeigersinn einmal umläuft.
Der Grund dafür war, dass der Weg jeweils mittels Homotopie auf einen
kreisformigen Weg um den Punkt $z_0$ deformiert wurde, der den Punkt $z_0$
ebenfalls genau einmal gegen den Uhrzeigersinn umläuft.

%
% Die Umlaufzahl eine Weges um einen Punkt
%
\subsubsection{Die Umlaufzahl eines Weges um einen Punkt}
Offenbar ist es notwendig, die Zahl der Umläufe eines Weges $\gamma$
im Gegenuhrzeigersinn um den Punkt $z_0$ etwas genauer zu fassen.
Damit diese Zahl wohldefiniert ist, darf der Weg den Punkt $z_0$
nicht treffen.
Wir zeichnen vom Punkt $z_0$ aus einen Strahl $s$ in eine beliebige
Richtung.
Schneidet die Kurve diesen Strahl nicht, dann lässt sich die Kurve
in der Menge  $\mathbb{C}\setminus s$ zu einem Punkt zusammeziehen,
insbesondere umläuft die Kurve den Punkt $z_0$ gar nicht.
Ein im Gegenurzeigersinn durchlaufener Kreis um den Punkt $z_0$
schneidet dagegen den Strahl genau einmal.
Durchläuft man den Kreis mehrmals, wie in $t\mapsto t+re^{2\pi int}$,
dann ist die Zahl der Schnittpunkte gerade $n$.
Die Anzahl der Kreuzungspunkte des Strahls mit der Kurve scheint
also an die Anzahl der Umläufe um den Punkt $z_0$ gekoppelt zu sein.
Dies funktioniert sogar für negative $n$, wenn man die Umläufe im
Uhrzeigersinn negativ zählt.

Die Anzahl der Schnittpunkte kann aber sogar unendlich gross werden,
wenn der Weg $\gamma$ ein Stück weit entlang eines Strahls verläuft.
Es ist auch mögich, dass der Weg von einer Seite auf den Strahl trifft,
ihn dann aber gleich wieder auf die selbe Seite verlässt.
In diesem Fall berührt der Weg den Strahl.
Der Strahl muss also so gewählt werden, dass der Weg ihn in jedem
Kreuzungspunkt mit einem positiven Schnittwinkel oder \emph{transversal}
\index{transversal}%
kreuzt.
Damit dies festgestellt werden kann, muss der Weg $\gamma$ differenzierbar
sein.
Durch Drehung des Strahls um den Punkt $z_0$ kann man immer erreichen,
dass der Weg den Strahl transversal schneidet.

Sei jetzt also $s$ ein von $z_0$ ausgehender Strahl, der den Weg 
$\gamma\colon I\to\mathbb{C}$ für die Punkte $t_i\in I$ schneidet.
\input{chapters/030-integration/fig/fig-argument.tex}%
Abbildung~\ref{buch:integration:fig:argument}
zeigt links den Wege $\gamma$ in rot, rechts das Argument von
$\gamma(t)-z_0$.
Wir zählen einen Kreuzungspunkt positiv, wenn das Argument 
\(
\varphi(t)
=
\operatorname{arg} (\gamma(t) - z_0)
\)
an der Stelle $t_i$ wächst, und negativ, wenn es abnimmt.
Wir setzen also
\begin{equation}
n_i
=
\begin{cases}
\phantom{-}1&\qquad \text{falls $\dot{\varphi}_i > 0$} \\
         - 1&\qquad \text{falls $\dot{\varphi}_i < 0$,} 
\end{cases}
\label{buch:integration:umlaufzahl:eqn:ni}
\end{equation}
wobei 
\[
v_i
=
\frac{d}{dt} \operatorname{arg}(\gamma(t)-z_0)\bigg|_{t=t_i}
\]
die Änderungsgeschwindigkeit des Arguments ist.
Die Umlaufzahl des Weges $\gamma$ um den Punkt $z_0$ ist dann die
Summe der $n_i$.

\begin{definition}[Umlaufzahl]
\label{buch:integration:definition:umlaufzahl}
\index{Umlaufzahl}%
Die \emph{Umlaufzahl} eines Weges $\gamma$ in der komplexen Ebene
$\mathbb{C}$ ist die Anzahl
\[
\operatorname{ind}_\gamma(z_0)
=
\sum_i n_i
\]
mit den Summanden $n_i$ aus
\eqref{buch:integration:umlaufzahl:eqn:ni}.
\end{definition}

In Abbildung~\ref{buch:integration:fig:argument} sind die Schnittpunkte
des Arguments mit der Richtung $\varphi_0$ des Strahls als orange
Punkte eingezeichnet.
Schnittpunkte, in denen das Argument ansteigt, werden positiv gezählt.
Schnittpunkte, in denen das Argument abnimmt, werden negativ gezählt.

%
% Umlaufzahl und komplexe Integration
%
\subsubsection{Umlaufzahl und komplexe Integration}
Die Umlaufzahl 
nach Definition~\ref{buch:integration:definition:umlaufzahl}
zu berechnen ist eher mühsam.
Daher ist es besonders erfreulich, dass man sie auch mit komplexer
Wegintegration berechnen kann, wie der folgende Satz zeigt.
Der Beweis beruht auf der Idee, dass sich der in der
Abbildung~\ref{buch:integration:fig:argument} rot gezeichnete Weg
in den grünen Weg deformieren lässt, der die Richtung $s$ immer in der
gleichen Richtung überschreitet und damit seinerseits homotop zu
einem Weg der Form $t\mapsto e^{2\pi int}$, für den sich das Integral
leicht berechnen lässt.

\begin{satz}
Sei $\gamma$ ein Weg in $\mathbb{C}$, der den Punkt $z_0$ nicht trifft,
dann umläuft der Weg den Punkt $z_0$
\[
\operatorname{ind}_\gamma(z_0)
=
\frac{1}{2\pi i}
\oint_\gamma \frac{dz}{z-z_0}
\]
mal.
\end{satz}

\begin{proof}
Abbildung~\ref{buch:integration:fig:argument} zeigt, wie die Umlaufzahl
berechnet wird.
Durch ``Strecken'' des Graphen von $\varphi(t)$ zu einer Geraden
(in der Abbildung grün eingezeichnet), kann der Weg durch ein
Homotopie in einen Weg deformiert werden, der durch  die
Parametrisierung $\gamma_1(t) = z_0+re^{2\pi i n t}$ beschrieben werden kann.
Nach Satz~\ref{buch:integration:satz:homotopie} ergeben die
die beiden Wege denselben Wert des Integrals.
Der Wert des Integrals über den Weg $\gamma_1$ ist
\[
\oint_{\gamma_1}
\frac{dz}{z-z_0}
=
\int_0^1
\frac{2\pi i nre^{2\pi int}}{z_0+re^{2\pi i nt}-z_0}
\,dt
=
\int_0^1
2\pi i n
\,dt
=
2\pi i n.
\]
Aufgelöst nach $n$ folgt daraus
\[
n
=
\operatorname{ind}_\gamma(z_0)
=
\frac{1}{2\pi i}
\oint_{\gamma_1}\frac{dz}{z-z_0}
=
\frac{1}{2\pi i}
\oint_{\gamma}\frac{dz}{z-z_0}
\]
Damit ist der Satz bewiesen.
\end{proof}

%
% Residuensatz für allgemein Wege
%
\subsubsection{Residuensatz für allgemeine Wege}
Der Residuensatz~\ref{buch:integration:satz:residuensatz} beruhte darauf,
dass der Weg $\gamma$ um den Punkt sich in einen Weg deformieren liess,
der sich entlang eines Kreises genau einmal im Gegenuhrzeigersinn um den
Punkt $z_0$ windet.
Die Umlaufzahl andererseits zeigt, dass sich Wege, die sich mehrmals
um den Punkt winden, ebenfalls deformieren lassen in einen Weg, der sich
entlang einer Kreislinie mit gleichmässig ansteigendem Argument gleichviele
Male um $z_0$ windet.

\begin{satz}
\label{buch:integration:satz:residuensatzn}
Sei $f\colon U \to \mathbb{C}$ eine holomorphe Funktion und $\gamma$
ein Weg in $U$, der den Punkt $z_0$ nicht trifft, aber in den Punkt $z_0$
zusammengezogen werden kann.
Dann gilt
\[
f(z_0)
\cdot
\operatorname{ind}_{\gamma}(z_0)
=
\frac{1}{2\pi i}
\oint_\gamma \frac{f(z)}{z-z_0}\,dz.
\]
\end{satz}

Die im Satz zusätzlich geforderte Bedingung, dass der Weg in den Punkt 
$z_0$ zusammengezogen werden kann, stellt sicher, dass sich der Weg nicht
um andere ``Löcher'' im Definitionsgebiet $U$ windet, in denen sich weitere
Polstellen der Funktion $f(z)$ verstecken könnten.
\input{chapters/030-integration/fig/fig-lochgebiet.tex}%
Sie ist in Abbildung~\ref{buch:integration:fig:lochgebiet} illustriert.
Der Weg $\gamma_1$ windet sich um das Loch im Gebiet $U$ herum und kann
daher nicht in einen Punkt zusammengezogen werden, dies ist nur für den
Weg $\gamma_2$ möglich.

%
% Ableitungen
%
\subsubsection{Ableitungen}
Auch für die Ableitungen lässt sich die Formel
\eqref{buch:integration:eqn:ableitungn}
verallgemeinern.
Mit einem Weg wie in Satz~\ref{buch:integration:satz:residuensatzn}
kann man die Ableitungen
\[
f^{(h)}(z_0)
\cdot
\operatorname{ind}_\gamma(z_0)
=
\frac{n!}{2\pi i}
\oint
\frac{f(z)}{(z-z_0)^n}
\,dz
\]
berechnen.
Alle Ableitungen lassen sich also allein aus den Werten der Funktion
$\gamma$ entlang des Weges $\gamma$ bestimmen.



